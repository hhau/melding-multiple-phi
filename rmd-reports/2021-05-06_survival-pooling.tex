% Options for packages loaded elsewhere
\PassOptionsToPackage{unicode,backref,colorlinks=true}{hyperref}
\PassOptionsToPackage{hyphens}{url}
%
\documentclass[
  10pt,
  a4paper,
]{article}
\usepackage{amsmath,amssymb}
\usepackage[]{tgpagella}
\usepackage{ifxetex,ifluatex}
\ifnum 0\ifxetex 1\fi\ifluatex 1\fi=0 % if pdftex
  \usepackage[T1]{fontenc}
  \usepackage[utf8]{inputenc}
  \usepackage{textcomp} % provide euro and other symbols
\else % if luatex or xetex
  \usepackage{unicode-math}
  \defaultfontfeatures{Scale=MatchLowercase}
  \defaultfontfeatures[\rmfamily]{Ligatures=TeX,Scale=1}
\fi
% Use upquote if available, for straight quotes in verbatim environments
\IfFileExists{upquote.sty}{\usepackage{upquote}}{}
\IfFileExists{microtype.sty}{% use microtype if available
  \usepackage[]{microtype}
  \UseMicrotypeSet[protrusion]{basicmath} % disable protrusion for tt fonts
}{}
\makeatletter
\@ifundefined{KOMAClassName}{% if non-KOMA class
  \IfFileExists{parskip.sty}{%
    \usepackage{parskip}
  }{% else
    \setlength{\parindent}{0pt}
    \setlength{\parskip}{6pt plus 2pt minus 1pt}}
}{% if KOMA class
  \KOMAoptions{parskip=half}}
\makeatother
\usepackage{xcolor}
\IfFileExists{xurl.sty}{\usepackage{xurl}}{} % add URL line breaks if available
\IfFileExists{bookmark.sty}{\usepackage{bookmark}}{\usepackage{hyperref}}
\hypersetup{
  pdftitle={Submodels, priors, and pooling the link parameters in the survival example},
  pdfauthor={Andrew Manderson},
  hidelinks,
  pdfcreator={LaTeX via pandoc}}
\urlstyle{same} % disable monospaced font for URLs
\usepackage[margin=2.25cm]{geometry}
\usepackage{graphicx}
\makeatletter
\def\maxwidth{\ifdim\Gin@nat@width>\linewidth\linewidth\else\Gin@nat@width\fi}
\def\maxheight{\ifdim\Gin@nat@height>\textheight\textheight\else\Gin@nat@height\fi}
\makeatother
% Scale images if necessary, so that they will not overflow the page
% margins by default, and it is still possible to overwrite the defaults
% using explicit options in \includegraphics[width, height, ...]{}
\setkeys{Gin}{width=\maxwidth,height=\maxheight,keepaspectratio}
% Set default figure placement to htbp
\makeatletter
\def\fps@figure{htbp}
\makeatother
\setlength{\emergencystretch}{3em} % prevent overfull lines
\providecommand{\tightlist}{%
  \setlength{\itemsep}{0pt}\setlength{\parskip}{0pt}}
\setcounter{secnumdepth}{5}
\usepackage{amsmath}
% I always seem to need tikz for something
\usepackage{tikz}
\usetikzlibrary{positioning, shapes, intersections, through, backgrounds, fit, decorations.pathmorphing, angles, quotes}
\usepackage{setspace}
\onehalfspacing

\usepackage{lineno}
% \linenumbers

\usepackage{relsize}
\usepackage{placeins}

% required for landscape pages. beware, they back the build very slow.
\usepackage{pdflscape}

% table - `gt' package uses these, often unimportant
\usepackage{longtable}
\usepackage{booktabs}
\usepackage{caption}

\usepackage{color}
\definecolor{myredhighlight}{RGB}{180, 15, 32}
\definecolor{mydarkblue}{RGB}{0, 33, 79}
\definecolor{mymidblue}{RGB}{44, 127, 184}
\definecolor{mylightblue}{RGB}{166, 233, 255}

\usepackage{colortbl}

\newcommand{\semitransp}[2][35]{\color{fg!#1}#2}

\setcounter{secnumdepth}{3}

\let\Oldcap\cap
\renewcommand{\cap}{\mathrel{\mathsmaller{\Oldcap}}}

% pd stands for: probability distribution and is useful to distringuish
% marignals for probabilities specifically p(p_{1}) and the like.
\newcommand{\pd}{\text{p}}
\newcommand{\q}{\text{q}}
\newcommand{\w}{\text{w}}
\newcommand{\pdr}{\text{r}}
\newcommand{\pdrh}{\hat{\text{r}}}

% melding
\newcommand{\ppoolphi}{\pd_{\text{pool}}(\phi)}
\newcommand{\pmeld}{\pd_{\text{meld}}}

% the q(x)w(x), "weighted target" density 
% for the moment I'm going to call it s(x), as that is the next letter of the 
% alphabet. Can change it later
\newcommand{\s}{\text{s}}
% direct density estimate - replaces lambda.
\newcommand{\ddest}{\text{s}}
% target weighting function
\newcommand{\tarw}{\text{u}}

% constants - usually sizes of things
\newcommand{\Nx}{N}
\newcommand{\Nnu}{\text{N}_{\text{nu}}}
\newcommand{\Nde}{\text{N}_{\text{de}}}
\newcommand{\Nmc}{\text{N}_{\text{mc}}}
\newcommand{\Nw}{W}
\newcommand{\Nm}{M}
\newcommand{\Ns}{S}
\newcommand{\Np}{P}

% locales - could switch to x and x'
\newcommand{\xnu}{x_{\text{nu}}}
\newcommand{\xde}{x_{\text{de}}}
\newcommand{\phinu}{\phi_{\text{nu}}}
\newcommand{\phide}{\phi_{\text{de}}}

% sugiyama stuff
\newcommand{\pdnu}{\pd_{\text{nu}}}
\newcommand{\pdde}{\pd_{\text{de}}}

% indices 
\newcommand{\wfindex}{w}
\newcommand{\sampleindex}{n}
\newcommand{\modelindex}{m}
\newcommand{\stageindex}{s}
\newcommand{\phiindex}{p}

% independence symbol
\newcommand{\indep}{\perp\!\!\!\perp}
\newcommand{\setcomp}{\mathsf{c}}

\newtheorem{theorem}{Theorem}[section]
\newtheorem{corollary}{Corollary}[theorem]

\DeclareMathOperator*{\argmin}{arg\,min}

% ARDS example in text commands
\newcommand{\paoii}{PaO\textsubscript{2}}
\newcommand{\fioii}{FiO\textsubscript{2}}
\newcommand{\spoii}{SpO\textsubscript{2}}
\newcommand{\pfratio}{\paoii/\fioii}
\newcommand{\sfratio}{\spoii/\fioii}
\ifluatex
  \usepackage{selnolig}  % disable illegal ligatures
\fi
\newlength{\cslhangindent}
\setlength{\cslhangindent}{1.5em}
\newlength{\csllabelwidth}
\setlength{\csllabelwidth}{3em}
\newenvironment{CSLReferences}[2] % #1 hanging-ident, #2 entry spacing
 {% don't indent paragraphs
  \setlength{\parindent}{0pt}
  % turn on hanging indent if param 1 is 1
  \ifodd #1 \everypar{\setlength{\hangindent}{\cslhangindent}}\ignorespaces\fi
  % set entry spacing
  \ifnum #2 > 0
  \setlength{\parskip}{#2\baselineskip}
  \fi
 }%
 {}
\usepackage{calc}
\newcommand{\CSLBlock}[1]{#1\hfill\break}
\newcommand{\CSLLeftMargin}[1]{\parbox[t]{\csllabelwidth}{#1}}
\newcommand{\CSLRightInline}[1]{\parbox[t]{\linewidth - \csllabelwidth}{#1}\break}
\newcommand{\CSLIndent}[1]{\hspace{\cslhangindent}#1}

\title{Submodels, priors, and pooling the link parameters in the
survival example}
\author{Andrew Manderson}
\date{08 June, 2021}

\begin{document}
\maketitle

\hypertarget{models}{%
\section{Models}\label{models}}

There are \(i = 1, \ldots, N\) individuals (icustays) in the data set.
Each individual is admitted to the ICU at time \(0\), and is discharged
or expires at time \(C_{i}\).

\hypertarget{pf-ratio-model-b-spline-pd_1}{%
\subsection{\texorpdfstring{P/F ratio model (B-spline):
\(\pd_{1}\)}{P/F ratio model (B-spline): \textbackslash pd\_\{1\}}}\label{pf-ratio-model-b-spline-pd_1}}

Each individual has \(j = 1, \ldots, J_{i}\) P/F ratio observations
\(z_{i, j}\) at times \(t_{i, j}\) where
\(\boldsymbol{z}_{i} = (z_{i, 1}, \ldots, z_{i, J_{i}})\) and
\(\boldsymbol{t}_{i} = (t_{i, 1}, \ldots, t_{i, J_{i}})\). We choose to
model the P/F ratio using a B-spline of degree 3, with 2 boundary knots
and 7 internal knots, and do not include an intercept column in the
spline basis. The lower boundary knot is placed at
\(\min(\boldsymbol{t_{i}})\) and the upper boundary knot is placed at
\(\max(\boldsymbol{t_{i}})\). The internal knots are evenly spaced
between the two boundary knots. These choices result in
\(k = 1, \ldots, 10\) spline basis terms per individual, with
coefficients \(\zeta_{i, k}\) where
\(\boldsymbol{\zeta}_{i} = (\zeta_{i, 1}, \ldots, \zeta_{i, 10})\). We
denote the individual specific B-spline basis evaluated at time
\(t_{i, j}\) as \(B_{i}(t_{i, j}) \in \mathbb{R}_{+} \cup \{0\}\).

An individual's P/F data are standardised to improve computational
performance. That is to say we actually observe \(\tilde{z}_{i, j}\),
which are then rescaled by each individual's mean \(\overline{z}_{i}\)
and standard deviation \(\hat{s}_{i}\) such that
\(z_{i, j} = \frac{\tilde{z}_{i, j} - \overline{z}_{i}}{\hat{s}_{i}}\).
The threshold for respiratory failure is rescaled for each individual,
i.e.~\(\tau_{i} = \frac{300 - \overline{z}_{i}}{\hat{s}_{i}}\).

We write the submodel \begin{equation}
\begin{gathered}
  z_{i, j} = \beta_{0, i} + B_{i}(t_{i, j})\boldsymbol{\zeta}_{i} + \varepsilon_{i, j} \\
  \beta_{0, i} \sim \text{N}(0, 1^2), \,\, \varepsilon_{i, j} \sim t_{5}(0, \omega), \,\,  \omega \sim \text{N}_{+}(0, 1^2).
\end{gathered}
\end{equation} For the spline basis coefficients we set
\(\zeta_{i, 1} \sim \text{N}(0, 0.5^2)\), and for \(k = 2, \ldots, 10\)
we employ a random-walk prior
\(\zeta_{i, k} \sim \text{N}(\zeta_{i, k - 1}, 0.5^2)\).

Individuals experience either the respiratory failure event
(\(d_{i} = 1\)) or are censored (\(d_{i} = 0\)) at time \(C_{i}\). The
event time \(T_{i}\) is defined as the solution to the following
optimisation problem \begin{equation}
  T_{i} = \min_{t} \left\{
    \tau_{i} = \beta_{0, i} + B_{i}(t)\boldsymbol{\zeta}_{i}
    \mid
    t \in [\max(0, \min(\boldsymbol{t_{i}})), \max(\boldsymbol{t_{i}})]
  \right\},
  \label{eqn:event_time_model_def}
\end{equation} where solutions to
\(\tau_{i} = \beta_{0, i} + B_{i}(t)\boldsymbol{\zeta}_{i}\) are
obtained using the multiple root finder of
\protect\hyperlink{ref-soetaert_rootsolve_2020}{Soetaert \emph{et al.}}
(\protect\hyperlink{ref-soetaert_rootsolve_2020}{2020}). If no solutions
are found then we assume the individual is censored, and set
\(T_{i} = C_{i}\) and \(d_{i} = 0\).

We define \(\phi_{1 \cap 2} = (\{T_{i}, d_{i}\}_{i = 1}^{N})\) noting
that
\(\pd_{1}(\phi_{1 \cap 2}) = \prod_{i = 1}^{N}\pd_{1, i}(T_{i}, d_{i})\).
It is also important to note that \(\pd_{1, i}(T_{i}, d_{i})\)
conditions on each individual's length of stay (in specifying the
location of the knots), as well as the range, mean, and standard
deviation of the P/F data (by standardising the \(\tilde{z}_{i, j}\)).
The analytic form of \(\pd_{1, i}(T_{i}, d_{i})\) is not available and
must be estimated, which we discuss in Section
\ref{estimating-submodel-prior-marginal-distributions}.

To align with our chained melding notation we define
\(Y_{1} = (\{\boldsymbol{z}_{i}, \boldsymbol{t}_{i}\}_{i = 1}^{N})\) and
\(\psi_{1} = (\{\beta_{0, i}, \boldsymbol{\zeta}_{i}\}_{i = 1}^{N}, \omega)\).

\hypertarget{cumulative-fluid-model-piecewise-linear-pd_3}{%
\subsection{\texorpdfstring{Cumulative fluid model (piecewise linear)
\(\pd_{3}\)}{Cumulative fluid model (piecewise linear) \textbackslash pd\_\{3\}}}\label{cumulative-fluid-model-piecewise-linear-pd_3}}

Each individual has \(l = 1, \ldots, L_{i}\) 24-hourly fluid balance
observations\footnote{Details about the derivation of these values from
  the raw fluid data are contained in Appendix
  \ref{calculating-the-cumulative-fluid-balance-from-the-raw-fluid-data}.},
which are used to compute the cumulative fluid balance data \(x_{i, l}\)
with with \(\boldsymbol{x}_{i} = (x_{i, 1}, \ldots, x_{i, L_{i}})\).
These data are measured in litres and observed at times \(u_{i, l}\)
with \(\boldsymbol{u}_{i} = (u_{i, 1}, \ldots, u_{i, L_{i}})\). We use a
piecewise linear model for these data, with with slope
\(\eta_{1, i}^{b}\) before the breakpoint at time \(\kappa_{i}\), and
slope \(\eta_{1, i}^{a}\) after the breakpoint. Mathematically,
\begin{equation}
\begin{gathered}
  x_{i, l} = \eta_{0, i} + \eta^{b}_{1, i}(u_{i, l} - \kappa_{i})\boldsymbol{1}_{\{u_{i, l} < \kappa_{i}\}} + \eta^{a}_{1, i}(u_{i, l} - \kappa_{i})\boldsymbol{1}_{\{u_{i, l} \geq \kappa_{i}\}} + \epsilon_{i, l} \\
  \eta^{b}_{1, i} \sim \text{Gamma}(1.53, 0.24), \,\, \eta^{a}_{1, i} \sim \text{Gamma}(1.53, 0.24), \\
  \epsilon_{i, l} \sim \text{N}(0, \sigma^{2}_{x}),  \,\, \sigma_{x} \sim \text{N}_{+}(0, 5^2).
  \label{eqn:piecewise-fluid-model}
\end{gathered}
\end{equation}

The parameters for the gamma prior for \(\eta^{b}_{1, i}\) and
\(\eta^{a}_{1, i}\) are obtained by assuming that the 2.5-, 50-, and
97.5- percentiles are at 0.5, 5, and 20
(\protect\hyperlink{ref-belgorodski_rriskdistributions_2017}{Belgorodski
\emph{et al.}, 2017}). A slope of \(0.5\) (i.e.~the change in cumulative
fluid balance per day) is unlikely but possible due to missing data; a
slope of \(20\) is also unlikely but possible as extremely unwell
patients can have very high respiratory rates and thus require large
fluid inputs.

The prior for the breakpoint \(\kappa_{i}\) is derived as follows.
Define \(u_{i, (1)} = \min(\boldsymbol{u}_{i})\) and
\(u_{i, (n)} = \max(\boldsymbol{u}_{i})\), with
\(r_{i} = u_{i, (n)} - u_{i, (1)}\). We reparameterise the breakpoint by
noting that \(\kappa_{i} = \kappa^{\text{raw}}_{i}r_{i} + u_{i, (1)}\),
where \(\kappa^{\text{raw}} \in [0, 1]\). We then set
\(\kappa^{\text{raw}}_{i} \sim \text{Beta}(5, 5)\) to regularise the
breakpoint towards the middle of each individual's stay in ICU. This is
crucial to ensure the submodel is identifiable when there is no evidence
of a breakpoint in the data.

Specifying a prior for \(\eta_{0, i}\) is difficult because
\(\eta_{0, i}\) is the cumulative fluid balance at \(\kappa_{i}\), and
thus depends on the length of stay. Instead, we reparameterise
\(\eta_{0, i}\) to be a function of the y-intercept
\(\eta_{0, i}^{\text{raw}}\) \begin{equation}
  \eta_{0, i} =
    (\eta_{0, i}^{\text{raw}} + \eta^{b}_{1, i} \kappa_{i}) \boldsymbol{1}_{\{0 < \kappa_{i}\}} +
    (\eta_{0, i}^{\text{raw}} + \eta^{a}_{1, i} \kappa_{i}) \boldsymbol{1}_{\{0 \geq \kappa_{i}\}}.
\end{equation} We place a \(\text{LogNormal}(1.61, 0.47^2)\) prior on
\(\eta_{0, i}^{\text{raw}}\). These values are obtained assuming that, a
priori, the \(2.5\%, 50\%\), and \(99\%\) percentiles of
\(\eta_{0, i}^{\text{raw}}\) are \(0.5, 5\), and \(15\) respectively
(\protect\hyperlink{ref-belgorodski_rriskdistributions_2017}{Belgorodski
\emph{et al.}, 2017}). This is a broad prior that reflects the numerous
possible admission routes into the ICU. We expect those admitted from
the wards to have little pre-admission fluid data. Those admitted from
the operating theatre often have their in-theatre fluid input recorded
after admission into the ICU, with no easy way to distinguish these
records in the data.

To align with our melding notation we define
\(m_{i}(t) = \eta_{0, i} + \eta^{b}_{1, i}(t - \kappa_{i})\boldsymbol{1}_{\{t < \kappa_{i}\}} + \eta^{a}_{1, i}(t - \kappa_{i})\boldsymbol{1}_{\{t \geq \kappa_{i}\}}\),
with
\(\phi_{2 \cap 3} = \left(\{\eta^{b}_{1, i}, \eta^{a}_{1, i}, \kappa_{i}\}_{i = 1}^{N}\right)\),
\(Y_{3} = (\{\boldsymbol{x}_{i}, \boldsymbol{u}_{i}\}_{i = 1}^{N})\),
and \(\psi_{3} = (\{\eta_{0, i}\}_{i = 1}^{N}, \sigma^{2}_{x})\). Note
that we have explicit, analytic priors for the components of
\(\phi_{2 \cap 3}\). Hence, \begin{equation}
  \pd_{3}(\phi_{2 \cap 3}) = \prod_{i = 1}^{N} \pd(\eta^{b}_{1, i}) \pd(\eta^{a}_{1, i}) \pd(\kappa_{i}), \,\, \text{with} \,\,\,
  \pd(\kappa_{i}) = \pd_{\kappa^{\text{raw}}_{i}}(\frac{\kappa_{i} - u_{i, (1)}}{r_{i}}) \frac{1}{r_{i}}
\end{equation} by the change of variables formula.

\hypertarget{competing-risks-survival-model-pd_2}{%
\subsection{\texorpdfstring{Competing risks survival model
\(\pd_{2}\)}{Competing risks survival model \textbackslash pd\_\{2\}}}\label{competing-risks-survival-model-pd_2}}

Individuals experience one of two competing events: respiratory failure
(\(d_{i} = 1\)), or death or discharge \((d_{i} = 2)\). The competing
risks framework (see
\protect\hyperlink{ref-kalbfleisch_statistical_2002}{Kalbfleisch and
Prentice, 2002} for an introduction) requires hazards that are specific
to each type of event. We use a Weibull hazard with shape parameter
\(\gamma_{1}\) for the respiratory failure (RF) event times, and assume
a constant hazard \(\gamma_{2}\) for the death or discharge (DD) event
times. Denote \(\boldsymbol{\gamma} = (\gamma_{1}, \gamma_{2})\). All
individuals have \(b = 1, \ldots, B\) baseline (time invariant)
covariates \(w_{i, b}\) with
\(\boldsymbol{w}_{i} = (1, w_{i, 1}, \ldots, w_{i, B})\)(i.e.~including
an intercept term), and coefficient \(\theta \in \mathbb{R}^{B + 1}\).
The RF hazard is assumed to be influenced by these covariates and the
rate of increase in the cumulative fluid balance. The strength of the
latter relationship is captured by \(\alpha\). Hence, the RF hazard is
\begin{gather}
  h_{i, 1}(T_{i}) = \gamma_{1} T_{i}^{\gamma_{1} - 1} \exp\left\{\boldsymbol{w}_{i}\theta + \alpha \frac{\partial}{\partial T_{i}} m_{i}(T_{i})\right\} \\
  \frac{\partial}{\partial T_{i}} m_{i}(T_{i}) = \eta^{b}_{1, i}\boldsymbol{1}_{\{T_{i} < \kappa_{i}\}} + \eta^{a}_{1, i}\boldsymbol{1}_{\{T_{i} \geq \kappa_{i}\}},
\end{gather} and the DD hazard is \(h_{i, 2}(T_{i}) = \gamma_{2}\). The
RF cumulative hazard \(H_{i, 1}(T_{i})\) is, for \(T_{i} > \kappa_{i}\),
\begin{equation}
  H_{i, 1}(T_{i})
  = \int_{0}^{T_{i}} h_{i, 1}(u) \text{d}u
  = \exp\{\boldsymbol{w}_{i}\theta\}
    \left[
      \exp\left\{
        \alpha \eta^{b}_{1, i}
      \right\}
      \kappa_{i}^{\gamma}
      +
      \exp\left\{
        \alpha \eta^{a}_{1, i}
      \right\}
      (T_{i}^{\gamma} - \kappa_{i}^{\gamma})
    \right],
\end{equation} and for \(T_{i} < \kappa_{i}\) \begin{equation}
  H_{i, 1}(T_{i})
  = \int_{0}^{T_{i}} h_{i}(u) \text{d}u
  = T_{i}^{\gamma} \exp\{\boldsymbol{w}_{i}\theta + \alpha \eta^{b}_{1, i}\}.
\end{equation} The hazard-specific survival functions are
\(S_{i, d_{i}}(T_{i}) = \exp\{-H_{i, d_{i}}(t)\}\), and the overall
survival probability is \begin{equation}
  S_{i}(T_{i}) = \exp\left\{-\sum_{d_{i} \in \{1, 2\}} H_{i, d_{i}}(T_{i})\right\}.
\end{equation} Because there are no censored event times, the likelihood
for this competing risks model is \begin{equation}
  \pd(T_{i}, d_{i} \mid \boldsymbol{\gamma}, \boldsymbol{\theta}, \alpha, \kappa_{i}, \eta_{1, i}^{b}, \eta_{1, i}^{a}, \boldsymbol{w}_{i}) = h_{i, d_{i}}(T_{i}) S_{i}(T_{i}), \\
\end{equation} where we suppress the dependence on the parameters on the
right hand side for brevity. Our priors are \begin{equation}
  \boldsymbol{\gamma} \sim \text{N}_{+}(0, \boldsymbol{I}_{2}), \, \,
  \theta_{1} \sim \text{N}(\hat{E}, 1^2), \, \,
  (\theta_{2}, \ldots, \theta_{B + 1}) \sim \text{N}(0, \boldsymbol{I}_{B}), \, \,
  \alpha \sim \text{N}(0, 1^2),
  \label{eqn:surv-submodel-prior-def}
\end{equation} where \(\hat{E}\) is the log of the crude event rate
(\protect\hyperlink{ref-brilleman_bayesian_2020}{Brilleman \emph{et
al.}, 2020}), and \(\boldsymbol{I}_{p}\) is the \(p \times p\) identity
matrix. We adopt the same priors as the cumulative fluid balance
submodel for \(\kappa_{i}, \eta_{1, i}^{b}\), and \(\eta_{1, i}^{a}\).

\hypertarget{estimating-submodel-prior-marginal-distributions}{%
\section{Estimating submodel prior marginal
distributions}\label{estimating-submodel-prior-marginal-distributions}}

We must estimate \(\pd_{1}(\phi_{1 \cap 2})\) and
\(\pd_{2}(\phi_{1 \cap 2}, \phi_{2 \cap 3})\). Because these
distributions are functions of discrete and continuous parameters, and
some of the latter have different supports, standard kernel density
estimation as suggested by
\protect\hyperlink{ref-goudie_joining_2019}{Goudie \emph{et al.}}
(\protect\hyperlink{ref-goudie_joining_2019}{2019}) is inappropriate.
Instead we fit appropriate parametric mixture distributions using Monte
Carlo samples from the priors.

\hypertarget{pf-submodel}{%
\subsection{PF submodel}\label{pf-submodel}}

We approximate \(\pd_{1}(\phi_{1 \cap 2})\) using spike-and-slab mixture
distributions, with a spike at \(C_{i}\) for the censored events and a
beta distribution for the rescaled event times. Monte Carlo samples of
\(T_{i}\) and \(d_{i}\) are obtained from \(\pd_{1, i}(T_{i}, d_{i})\)
by drawing \(\beta_{0, i}\) and \(\boldsymbol{\zeta}_{i}\) from their
respective prior distributions and then solving
\eqref{eqn:event_time_model_def}. Mathematically, \begin{equation}
  \widehat{\pd}_{1}(T_{i}, d_{i}) =
    \widehat{\pi}_{i} \, \text{Beta}\left(\frac{T_{i}}{C_{i}}; \widehat{a}, \widehat{b}\right) \frac{1}{C_{i}} \boldsymbol{1}_{\{d_{i} = 1\}} +
    (1 - \widehat{\pi}_{i}) \boldsymbol{1}_{\{d_{i} = 2, T_{i} = C_{i}\}}
  \label{eqn:pf-event-time-prior-dist}
\end{equation} where \(\widehat{\pi}_{i}, \widehat{a}_{i}\) and
\(\widehat{b}_{i}\) are maximum likelihood estimates obtained using the
prior predictive samples. Estimates of \(\pd_{1, i}(T_{i}, d_{i})\) are
displayed in Figure \ref{fig:pf_prior_fit}.

\begin{figure}

{\centering \includegraphics{../plots/mimic-example/pf-prior-plot-small} 

}

\caption{Fitted distribution (curve) and Monte Carlo samples drawn from the prior (histogram) for a subset of the individuals in the cohort. The height of the atom at $C_{i}$ (red bar) has been set to $1 - \widehat{\pi}_{i}$}\label{fig:pf_prior_fit}
\end{figure}

\hypertarget{survival-submodel}{%
\subsection{Survival submodel}\label{survival-submodel}}

Our estimate of \(\pd_{2}(\phi_{1 \cap 2}, \phi_{2 \cap 3})\) relies on
the fact that
\(\pd_{2}(\phi_{1 \cap 2}, \phi_{2 \cap 3}) = \prod_{i = 1}^{N}\pd_{2, i}(T_{i}, d_{i}, \kappa_{i}, \eta^{b}_{1, i}, \eta^{a}_{1, i})\).
We focus on estimating
\(\pd_{2, i}(T_{i}, d_{i}, \kappa_{i}, \eta^{b}_{1, i}, \eta^{a}_{1, i})\)
for each individual and taking the product of these estimates.

Drawing samples from
\(\pd_{2, i}(T_{i}, d_{i}, \kappa_{i}, \eta^{b}_{1, i}, \eta^{a}_{1, i})\)
is a non-trivial task. We use the methodology of
\protect\hyperlink{ref-crowther_simulating_2013}{Crowther and Lambert}
(\protect\hyperlink{ref-crowther_simulating_2013}{2013}) as implemented
in \protect\hyperlink{ref-brilleman_simulate_2021}{Brilleman}
(\protect\hyperlink{ref-brilleman_simulate_2021}{2021}).

An inspection of the samples reveals correlation between
\((T_{i}, d_{i})\) and
\((\kappa_{i}, \eta^{b}_{1, i}, \eta^{a}_{1, i})\) that we would like to
capture in our estimate. To do so, we define transformed versions of the
continuous parameters, which have support on \(\mathbb{R}\) (which we
call the unconstrained parameters) \begin{equation}
  \tilde{T}_{i} = \text{Logit}\left(\frac{T_{i}}{C_{i}}\right), \quad
  \tilde{\kappa}_{i} = \text{Logit}\left(\frac{\kappa_{i} - u_{i, (1)}}{u_{i, (n)} - u_{i, (1)}}\right), \quad
  \tilde{\eta}^{b}_{1, i} = \log(\eta^{b}_{1, i}), \quad
  \tilde{\eta}^{a}_{1, i} = \log(\eta^{a}_{1, i})
\end{equation}

\hypertarget{other-details}{%
\section{Other details}\label{other-details}}

\begin{itemize}
\tightlist
\item
  log pooling info
\item
  multi-stage sampler info

  \begin{itemize}
  \tightlist
  \item
    chains / iterations / diagnostics
  \end{itemize}
\end{itemize}

\hypertarget{competing-risk-model-results}{%
\section{Competing risk model
results}\label{competing-risk-model-results}}

These are similar to before, but there is less difference between the
method

\begin{center}\includegraphics{../plots/mimic-example/psi-2-method-comparison-small} \end{center}

\hypertarget{pooling-the-priors}{%
\section{Pooling the priors}\label{pooling-the-priors}}

Once we have expressions or estimates of the prior marginal
distributions, we need to:

\begin{itemize}
\tightlist
\item
  Decide what type of pooling to use (probably all of them and compare)
\item
  Choose pooling weights, and run some kind of prior predictive check?
\item
  Apportion the pooled prior over the stages of the multi-stage sampler,
  and which stage to divide by the marginals.
\end{itemize}

\hypertarget{bibliography}{%
\section{Bibliography}\label{bibliography}}

\hypertarget{refs}{}
\begin{CSLReferences}{1}{0}
\leavevmode\hypertarget{ref-belgorodski_rriskdistributions_2017}{}%
Belgorodski, N., Greiner, M., Tolksdorf, K., et al. (2017)
{rriskDistributions}: {Fitting} distributions to given data or known
quantiles.

\leavevmode\hypertarget{ref-brilleman_simulate_2021}{}%
Brilleman, S. (2021) Simulate {Survival Data} {[}{R} package simsurv
version 1.0.0{]}. https://CRAN.R-project.org/package=simsurv;
{Comprehensive R Archive Network (CRAN)}.

\leavevmode\hypertarget{ref-brilleman_bayesian_2020}{}%
Brilleman, S. L., Elci, E. M., Novik, J. B., et al. (2020) Bayesian
survival analysis using the rstanarm {R} package. \emph{arXiv:2002.09633
{[}stat{]}}. Available at: \url{http://arxiv.org/abs/2002.09633}.

\leavevmode\hypertarget{ref-crowther_simulating_2013}{}%
Crowther, M. J. and Lambert, P. C. (2013) Simulating biologically
plausible complex survival data. \emph{Statistics in Medicine},
\textbf{32}, 4118--4134. DOI:
\href{https://doi.org/10.1002/sim.5823}{10.1002/sim.5823}.

\leavevmode\hypertarget{ref-goudie_joining_2019}{}%
Goudie, R. J. B., Presanis, A. M., Lunn, D., et al. (2019) Joining and
splitting models with {Markov} melding. \emph{Bayesian Analysis},
\textbf{14}, 81--109. DOI:
\href{https://doi.org/10.1214/18-BA1104}{10.1214/18-BA1104}.

\leavevmode\hypertarget{ref-kalbfleisch_statistical_2002}{}%
Kalbfleisch, J. D. and Prentice, R. L. (2002) \emph{The Statistical
Analysis of Failure Time Data}. 2nd ed. Wiley series in probability and
statistics. {Hoboken, N.J}: {J. Wiley}.

\leavevmode\hypertarget{ref-soetaert_rootsolve_2020}{}%
Soetaert, K., Hindmarsh, A. C., Eisenstat, S. C., et al. (2020)
{rootSolve}: {Nonlinear Root Finding}, {Equilibrium} and {Steady}-{State
Analysis} of {Ordinary Differential Equations}.

\end{CSLReferences}

\renewcommand{\thesection}{\Alph{section}}
\setcounter{section}{0}

\hypertarget{a-comparison-with-the-competing-risk-approach}{%
\section{A comparison with the competing risk
approach}\label{a-comparison-with-the-competing-risk-approach}}

An initially appealing approach is to consider a competing risks model
for \(\pd_{2}\), where each individual experiences either the
respiratory failure event or the competing, non-independent event of
death or discharge. However, issues arise due to the need to practically
align the supports of \(\pd_{1}(\phi_{1 \cap 2})\) and
\(\pd_{2}(\phi_{1 \cap 2})\), which requires conditioning on \(C_{i}\)
(the length of stay) in \(\pd_{2}\). Conditional on \(C_{i}\), the death
or discharge event can only happen at a known, fixed time, which
violates the competing risk assumption (that each event can occur at any
moment in time the individual is exposed to both risks). In light of
this, we feel that it is more correct to consider the time of death or
discharge as a censoring time. Standard survival analyses arguments tell
us that these approaches are equivalent. However, these arguments make
the key assumption that the survival times and indicators
\((T_{i}, d_{i})\) are known quantities. This assumption is not valid in
our example, and below we show why this invalidates the usual
equivalence between the competing risk and censoring approaches.

Suppose that each individual \(i\) experiences one of \(d_{i} = 1, 2\)
competing risks. We observe \(\{T_{i}, d_{i}\}\), where \(d_{i} = 1\)
indicates that individual \(i\) experienced respiratory failure at time
\(T_{i}\). If \(d_{i} = 2\) then individual \(i\) expired or was
discharged at time \(T_{i}\), noting that this event must occur at time
\(C_{i}\). Each cause-specific hazard has parameters \(\theta_{d_{i}}\)
and we denote the hazard
\(h_{i, d_{i}}(t \mid \theta_{d_{i}}, \boldsymbol{w}_{i})\). Denote
\(\boldsymbol{\theta} = (\theta_{1}, \theta_{2})\) and assume only one
such event can occur at a time so that \begin{gather}
  h_{i}(T_{i} \mid \boldsymbol{\theta}, \boldsymbol{w}_{i}) = \sum_{d_{i} \in \{1, 2\}} h_{i, d_{i}}(T_{i} \mid \theta_{d_{i}}, \boldsymbol{w}_{i}), \\
  \begin{aligned}
  H_{i}(T_{i} \mid \boldsymbol{\theta}, \boldsymbol{w}_{i})
    &= \int_{0}^{T_{i}} \sum_{d_{i} \in \{1, 2\}} h_{i, d_{i}}(u \mid \theta_{d_{i}}, \boldsymbol{w}_{i}) \text{d}u \\
    &= \sum_{d_{i} \in \{1, 2\}} \int_{0}^{T_{i}} h_{i, d_{i}}(u \mid \theta_{d_{i}}, \boldsymbol{w}_{i}) \text{d}u \\
    &= \sum_{d_{i} \in \{1, 2\}} H_{i, d_{i}}(T_{i} \mid \theta_{d_{i}}, \boldsymbol{w}_{i}),
  \end{aligned} \\
  S_{i}(T_{i} \mid \boldsymbol{\theta}, \boldsymbol{w}_{i})
    = \exp\left\{-H_{i}(T_{i} \mid \boldsymbol{\theta}, \boldsymbol{w}_{i})\right\}
    = \exp\left\{-\sum_{d_{i} \in \{1, 2\}} H_{i, d_{i}}(T_{i} \mid \theta_{d_{i}}, \boldsymbol{w}_{i})\right\}.
\end{gather} As per Equation (8.8) in
\protect\hyperlink{ref-kalbfleisch_statistical_2002}{Kalbfleisch and
Prentice} (\protect\hyperlink{ref-kalbfleisch_statistical_2002}{2002})
the likelihood function for a specific individual is \begin{align*}
  \pd(T_{i}, d_{i} \mid \boldsymbol{\theta}, \boldsymbol{w}_{i})
    &= h_{i, d_{i}}(T_{i} \mid \theta_{d_{i}}, \boldsymbol{w}_{i}) S_{i}(T_{i} \mid \boldsymbol{\theta}, \boldsymbol{w}_{i}) \\
    &= h_{i, d_{i}}(T_{i} \mid \theta_{d_{i}}, \boldsymbol{w}_{i}) \exp\left\{-\sum_{d_{i} \in \{1, 2\}} H_{i, d_{i}}(T_{i} \mid \theta_{d_{i}}, \boldsymbol{w}_{i})\right\}.
\end{align*}

It is now necessary to assume

\begin{itemize}
\tightlist
\item
  that there are no shared elements in \(\theta_{1}\) and \(\theta_{2}\)
  and they are a priori independent,
\item
  that \(\theta_{2}\) is not of interest, i.e.~we wish to
  integrate/marginalise \(\theta_{2}\) out of the likelihood.
\end{itemize}

The model (given covariates \(\boldsymbol{w}_{i}\)) is

\begin{equation}
  \pd(T_{i}, d_{i}, \boldsymbol{\theta} \mid \boldsymbol{w}_{i}) =
    \pd(T_{i}, d_{i} \mid \boldsymbol{\theta}, \boldsymbol{w}_{i})\pd(\boldsymbol{\theta}).
\end{equation}

We are interested in the following marginal

\begin{equation}
  \pd(T_{i}, d_{i}, \theta_{1} \mid \boldsymbol{w}_{i})
  = \int \pd(T_{i}, d_{i}, \boldsymbol{\theta} \mid \boldsymbol{w}_{i}) \text{d}\theta_{2}
  = \int h_{i, d_{i}}(T_{i} \mid \theta_{d_{i}}, \boldsymbol{w}_{i}) S_{i}(T_{i} \mid \boldsymbol{\theta}, \boldsymbol{w}_{i}) \pd(\theta_{1}) \pd(\theta_{2}) \text{d}\theta_{2}.
\end{equation} If \(d_{i} = 1\) \begin{equation}
  \pd(T_{i}, d_{i}, \theta_{1} \mid \boldsymbol{w}_{i})
  = h_{i, 1}(T_{i} \mid \theta_{1}, \boldsymbol{w}_{i}) S_{i, 1}(T_{i} \mid \theta_{1}, \boldsymbol{w}_{i}) \pd(\theta_{1}) \int S_{i, 2}(T_{i} \mid \theta_{2}, \boldsymbol{w}_{i}) \pd(\theta_{2}) \text{d} \theta_{2},
  \label{eqn:competing-risks-deriv-one}
\end{equation} and if \(d_{i} = 2\) \begin{equation}
  \pd(T_{i}, d_{i}, \theta_{1} \mid \boldsymbol{w}_{i})
  = S_{i, 1}(T_{i} \mid \theta_{1}, \boldsymbol{w}_{i}) \pd(\theta_{1}) \int h_{i, 2}(T_{i} \mid \theta_{2}, \boldsymbol{w}_{i}) S_{i, 2}(T_{i} \mid \theta_{2}, \boldsymbol{w}_{i}) \pd(\theta_{2}) \text{d} \theta_{2}.
  \label{eqn:competing-risks-deriv-two}
\end{equation} Standard survival analyses consider \(T_{i}\) as data.
Under this assumption the integrals in
\eqref{eqn:competing-risks-deriv-one} and
\eqref{eqn:competing-risks-deriv-two} are constants that do not depend
on the parameters of interest, and can be ignored when maximising the
likelihood for \(\theta_{1}\). The remaining components of
\eqref{eqn:competing-risks-deriv-one} and
\eqref{eqn:competing-risks-deriv-two} comprise the likelihood that would
be obtained by considering all non \(d_{i} = 1\) events as censored.
However, in our case \(T_{i}\) is a parameter, and hence the integrals
are non-ignorable functions of \(T_{i}\). This implies that the
censoring model and the competing risks model are not equivalent, which
we see in practice when comparing the posterior distributions for
\(\theta_{1}\) under both models.

\hypertarget{calculating-the-cumulative-fluid-balance-from-the-raw-fluid-data}{%
\section{Calculating the cumulative fluid balance from the raw fluid
data}\label{calculating-the-cumulative-fluid-balance-from-the-raw-fluid-data}}

In the raw fluid data each patient has
\(\tilde{l} = 1, \ldots, \tilde{L}_{i}\) observations. Each observation
\(\tilde{x}_{i, \tilde{l}}\) is typically a small fluid administration
(e.g.~an injection of some medicine in saline solution), or a fluid
discharge (almost always urine excretion). The observations have
corresponding observation times \(\tilde{u}_{i, \tilde{l}}\), with
\(\tilde{\boldsymbol{u}}_{i} = \{\tilde{u}_{i, 1}, \ldots, \tilde{u}_{i, \tilde{L}_{i}}\}\)
and
\(\tilde{\boldsymbol{x}}_{i} = \{\tilde{x}_{i, 1}, \ldots, \tilde{x}_{i, \tilde{L}_{i}}\}\).
We code the fluid administrations/inputs as positive values, and the
excretions/outputs as negative values. Each patient has an enormous
number of raw fluid observations \((L_{i} \ll \tilde{L}_{i})\) and it is
computationally infeasible to consider them all at once. Because the
time scale of interest is typically a number of days, we aggregate the
raw fluid observations into 24-hourly changes in fluid balance. From
these 24-hourly changes we calculate the cumulative fluid balance.

Mathematically, we define an ordered integer-vector of boundary values
\begin{equation}
  \boldsymbol{b}_{i} = (\lfloor \min\{\tilde{\boldsymbol{u}}_{i}\} \rfloor,  \lfloor \min\{\tilde{\boldsymbol{u}}_{i}\} \rfloor + 1, \ldots, \lceil \max\{\tilde{\boldsymbol{u}}_{i}\} \rceil),
\end{equation} noting that \(\dim(\boldsymbol{b}_{i}) = L_{i} + 1\). The
raw fluid observations are then divided up into \(L_{i}\) subsets of
\(\{\tilde{\boldsymbol{x}}_{i}, \tilde{\boldsymbol{u}}_{i}\}\) based on
which boundary values the observation falls in between: \begin{equation}
  B_{i, l} = \left\{
    \{\tilde{\boldsymbol{x}}_{i}, \tilde{\boldsymbol{u}}_{i}\}
    \mid
    b_{i, l} \leq \tilde{\boldsymbol{u}}_{i} < b_{i, l + 1}
  \right\},
\end{equation} for \(l = 1, \ldots, L_{i}\). Denote
\(N^{B}_{i, l} = \lvert B_{i, l} \rvert \mathop{/} 2\) (dividing by two
as \(B_{i, l}\) contains both the observation and the observation time).
The \(l\)\textsuperscript{th} 24-hourly fluid change \(\Delta_{i, l}\)
and corresponding observation time \(u_{i, l}\) can then be computed as
\begin{equation}
  \Delta_{i, l} = \sum_{s = 1}^{N^{B}_{i, l}} \tilde{x}_{i, s}, \,\, \text{s.t.} \,\, \tilde{x}_{i, s} \in B_{i, l}, \qquad
  u_{i, l} = \frac{1}{N^{B}_{i, l}} \sum_{s = 1}^{N^{B}_{i, l}} \tilde{u}_{i, s}, \,\, \text{s.t.} \,\, \tilde{u}_{i, s} \in B_{i, l}.
\end{equation} Finally, the 24-hourly cumulative fluid balance data are
computed by \(x_{i, l} = \sum_{s = 1}^{l} \Delta_{i, s}\), and we assume
they too are observed at \(u_{i, l}\).

\hypertarget{cohort-selection-criteria}{%
\section{Cohort selection criteria}\label{cohort-selection-criteria}}

\end{document}
