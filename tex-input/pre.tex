\usepackage{amsmath}
% I always seem to need tikz for something
\usepackage{tikz}
\usetikzlibrary{positioning, shapes, intersections, through, backgrounds, fit, decorations.pathmorphing, angles, quotes}
\usepackage{lineno}
% \linenumbers


\makeatletter
\@ifclassloaded{imsart}{}{
\usepackage{setspace}
\onehalfspacing
}
\makeatother

\usepackage{relsize}
\usepackage{placeins}

% required for landscape pages. beware, they back the build very slow.
\usepackage{pdflscape}

% table - `gt' package uses these, often unimportant
\usepackage{longtable}
\usepackage{booktabs}
\usepackage{caption}

\usepackage{color}
\definecolor{myredhighlight}{RGB}{180, 15, 32}
\definecolor{mydarkblue}{RGB}{0, 33, 79}
\definecolor{mymidblue}{RGB}{44, 127, 184}
\definecolor{mylightblue}{RGB}{166, 233, 255}

\usepackage{colortbl}

\newcommand{\semitransp}[2][35]{\color{fg!#1}#2}

\setcounter{secnumdepth}{3}

\let\Oldcap\cap
\renewcommand{\cap}{\mathrel{\mathsmaller{\Oldcap}}}

% pd stands for: probability distribution and is useful to distringuish
% marignals for probabilities specifically p(p_{1}) and the like.
\newcommand{\pd}{\text{p}}
\newcommand{\q}{\text{q}}
\newcommand{\w}{\text{w}}
\newcommand{\pdr}{\text{r}}
\newcommand{\pdrh}{\hat{\text{r}}}

% melding
\newcommand{\ppoolphi}{\pd_{\text{pool}}(\phi)}
\newcommand{\pmeld}{\pd_{\text{meld}}}

% the q(x)w(x), "weighted target" density 
% for the moment I'm going to call it s(x), as that is the next letter of the 
% alphabet. Can change it later
\newcommand{\s}{\text{s}}
% direct density estimate - replaces lambda.
\newcommand{\ddest}{\text{s}}
% target weighting function
\newcommand{\tarw}{\text{u}}

% constants - usually sizes of things
\newcommand{\Nx}{N}
\newcommand{\Nnu}{\text{N}_{\text{nu}}}
\newcommand{\Nde}{\text{N}_{\text{de}}}
\newcommand{\Nmc}{\text{N}_{\text{mc}}}
\newcommand{\Nw}{W}
\newcommand{\Nm}{M}
\newcommand{\Ns}{S}
\newcommand{\Np}{P}

% locales - could switch to x and x'
\newcommand{\xnu}{x_{\text{nu}}}
\newcommand{\xde}{x_{\text{de}}}
\newcommand{\phinu}{\phi_{\text{nu}}}
\newcommand{\phide}{\phi_{\text{de}}}

% sugiyama stuff
\newcommand{\pdnu}{\pd_{\text{nu}}}
\newcommand{\pdde}{\pd_{\text{de}}}

% indices 
\newcommand{\wfindex}{w}
\newcommand{\sampleindex}{n}
\newcommand{\modelindex}{m}
\newcommand{\stageindex}{s}
\newcommand{\phiindex}{p}

% independence symbol
\newcommand{\indep}{\perp\!\!\!\perp}
\newcommand{\setcomp}{\mathsf{c}}

\newtheorem{theorem}{Theorem}[section]
\newtheorem{corollary}{Corollary}[theorem]

\DeclareMathOperator*{\argmin}{arg\,min}

% ARDS example in text commands
\newcommand{\paoii}{PaO\textsubscript{2}}
\newcommand{\fioii}{FiO\textsubscript{2}}
\newcommand{\spoii}{SpO\textsubscript{2}}
\newcommand{\pfratio}{\paoii/\fioii}
\newcommand{\sfratio}{\spoii/\fioii}
